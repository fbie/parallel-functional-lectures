\documentclass{beamer}
\usepackage{beamerthemeOerestad}
\usetheme{Oerestad}

\usepackage[utf8]{inputenc}
\usepackage{amsfonts}
\usepackage{amsmath}
\usepackage{booktabs}
\usepackage{color}
\usepackage{colortbl}
\usepackage{dirtytalk}
\usepackage{graphicx}
\usepackage{hyperref}
\usepackage{listings}
\usepackage[square,sort,comma,numbers]{natbib}
\usepackage{url}
\usepackage{semantic}
\usepackage{tikz}
\usepackage{todonotes}

\hypersetup{
  colorlinks=true,
  urlcolor=cyan
}

\lstdefinestyle{Racket}{
  language=Lisp,
  basicstyle=\ttfamily,
  otherkeywords={\#lang,define,match,let,letrec,:,->,lambda,struct,define-type,car,cdr},
  identifierstyle=\color{pink!60!blue},
  keywordstyle=\color{blue!50!black},
  stringstyle=\color{green!50!black},
  commentstyle=\color{pink!60!black},
  showstringspaces=false
}

\lstdefinestyle{Java}{
  language=Java,
  basicstyle=\ttfamily,
  identifierstyle=\color{orange!40!black},
  keywordstyle=\color{blue!50!black},
  stringstyle=\color{green!50!black},
  commentstyle=\color{blue!70!gray},
  showstringspaces=false
}

\lstset{
  style=Java
}

\usetikzlibrary{shapes, shapes.multipart, chains, positioning}
\tikzset{cons/.style n args=2{
    on chain,
    rectangle split,
    rectangle split horizontal,
    rectangle split parts=2,
    draw,
    anchor=center,
    text height=1.5ex,
    node contents={#1\nodepart{two}#2},
    join={by ->}
  }
}

\title{Higher-Order Functional Programming\\with\\Java 8 Streams}
\author{Florian Biermann \\\small{\texttt{fbie@itu.dk}} \\~}

\institute{IT University of Copenhagen \& UCAS}

\date{2016-06-02}

\begin{document}

\begin{frame}
  \titlepage{}
\end{frame}

\begin{frame}{Today's Plan}
  We will be looking at (parallel) higher order functional programming in Java.

  \begin{itemize}
  \pause{} \item Brief review of Java's anonymous inner classes.
  \pause{} \item Java 8 lambda expressions.
  \pause{} \item Java 8 functional interfaces.
  \pause{} \item Higher-order functional programming using lazy streams.
  \end{itemize}
\end{frame}

\begin{frame}[fragile]{Java 7 Anonymous Inner Classes}

In Java, you can instantiate new classes on the fly:

\begin{lstlisting}
Runnable r =
  new Runnable() {
    public void run() {
      System.out.println("Inner class.");
    }};
r.run(); // "Inner class."
\end{lstlisting}

\pause{} \vspace{0.25cm}

They can access global variables iff these are marked \lstinline{final}:

\begin{lstlisting}
final String s = "This is final!";
Runnable r = new Runnable() {
    public void run() {
      System.out.println(s);
    }};
r.run(); // "This is final!";
\end{lstlisting}
\end{frame}

\begin{frame}[fragile]{Runnable Is Just an Interface}

\begin{lstlisting}
public interface Runnable {
  public void run();
}
\end{lstlisting}

\pause{} \vspace{0.25cm}

There are other useful interfaces, e.g. \lstinline{Callable<V>} from \lstinline{java.util.concurrent}:

\begin{lstlisting}
public interface Callable<V> {
  public V call();
}
\end{lstlisting}
\pause{}
\begin{lstlisting}
final String s = "Callable.";
Callable<String> c = new Callable<String>() {
    public String call() {
      return s;
    }};
System.out.println(c.call()); // "Callable."
\end{lstlisting}
\end{frame}

\begin{frame}[fragile]{Why Anonymous Inner Classes?}
  The Java answer to closures.

  \begin{itemize}
  \pause{} \item Whenever we want to start a thread, we pass it a new instance of \lstinline{Runnable}.
  \pause{} \item Short-running tasks, only used once.
  \pause{} \item Methods in Java are no \emph{first-class citizens}, but objects are.
  \pause{} \item Anonymous inner classes are a \emph{work-around} for this problem.
  \pause{} \item The syntax is ugly and hard to read, so Java 8 introduces \emph{lambda expressions.}
  \end{itemize}
\end{frame}

\begin{frame}[fragile]{Our First Java 8 Lambda Expression}

Equal to instantiating an anonymous \lstinline{Runnable}:
\begin{lstlisting}
Runnable r =
  () -> { System.out.println("Lambda."); } ;
\end{lstlisting}

\pause{} \vspace{0.5cm} No closing braces required if only one statement:
\begin{lstlisting}
Runnable r =
  () -> System.out.println("Lambda.");
\end{lstlisting}
\end{frame}

\begin{frame}[fragile]{Our Second Java 8 Lambda Expression}

Equal to instantiating an anonymous \lstinline{Callable<String>}:
\begin{lstlisting}
Callable<String> c =
  () -> { return "Lambda."; };
\end{lstlisting}

\pause{} \vspace{0.5cm} No \lstinline{return} statement required if only one statement:
\begin{lstlisting}
Callable<String> c = () -> "Lambda.";
\end{lstlisting}

\pause{} \vspace{0.5cm} Compare to Typed Racket, also no \lstinline{return} statement:
\begin{lstlisting}[style=Racket]
(lambda () "Lambda.")
\end{lstlisting}
\end{frame}

\begin{frame}[fragile]{This Is Illegal}
\begin{lstlisting}
Callable<String> c =
  () -> {
    System.out.println("Callable.call()");
    "Lambda.";
  };
\end{lstlisting}

\pause{}

{\color{red!70!black}
\begin{verbatim}
error: not a statement
        "Lambda.";};
        ^
\end{verbatim}}

\pause{}

\textbf{Rules:}
\begin{itemize}
  \pause{} \item If you use more than one statement then you must use curly braces.
  \pause{} \item If you use curly braces then you \emph{must} use \lstinline{return} (except for \lstinline{void} functions).
\end{itemize}
\end{frame}

\begin{frame}{Java Inner Classes Summary}
The important points until here:

\begin{itemize}
\pause{} \item Inner classes always implement an interface (or an abstract class).
\pause{} \item There are two important interfaces already defined in Java 7.
\pause{} \item In Java 8, you can use \emph{lambda expressions} instead of inner classes.
\pause{} \item But what about more interesting functions that also take parameter arguments?
\end{itemize}
\end{frame}

\begin{frame}[fragile]{Defining Your Own Functional Interfaces}
A function of type \lstinline{int} $->$ \lstinline{String}:
\begin{lstlisting}
@FunctionalInterface
public interface StringToInt {
  public int apply(String s);
}
\end{lstlisting}

\pause{} Now, we can instantiate such a function using a lambda expression:
\begin{lstlisting}
StringToInt parseToInt =
  s -> Integer.parseInt(s);
\end{lstlisting}

\begin{itemize}
\pause{} \item Lambda expressions \textbf{have no types themselves!}
\pause{} \item Instead, they have a \emph{target function type}.
\pause{} \item Without target function type, the lambda expression does not work.
\pause{} \item Here, one target function type is \lstinline{StringToInt}.
\end{itemize}
\end{frame}

\begin{frame}[fragile]{Java 8 Functional Interfaces}
No need to define your own interfaces all the time. \lstinline{java.util.function} contains \emph{many} different functional interfaces:

\pause{}\begin{lstlisting}
Function<String, String> toUpperCase =
  s -> s.toUpperCase();
\end{lstlisting}
\pause{}\begin{lstlisting}
Predicate<String> isEmpty =
  s -> s.isEmpty();
\end{lstlisting}
\pause{}\begin{lstlisting}
Consumer<String> print =
  s -> System.out.println(s);
\end{lstlisting}
\pause{}\begin{lstlisting}
Supplier<String> genString =
  () -> "Nihao.";
\end{lstlisting}
\end{frame}

\begin{frame}[fragile]{Primitive Data Type Interfaces}
To avoid \emph{value boxing} (\lstinline{int} $->$ \lstinline{Integer}), there are special implementations for primitive data types:

\pause{}\begin{lstlisting}
ToIntFunction<String> parseToInt =
  s -> Integer.parseInt(s);
\end{lstlisting}
\pause{}\begin{lstlisting}
LongBinaryOperator longMul =
  (x, y) -> x * y;
\end{lstlisting}
\pause{}\begin{lstlisting}
DoubleToIntFunction round =
  x -> (int)(x + 0.5d);
\end{lstlisting}
\pause{}\begin{lstlisting}
IntPredicate isEven =
  x -> x % 2 == 0;
\end{lstlisting}
\end{frame}

\begin{frame}[fragile]{Function References}
This is redundant syntax:
\begin{lstlisting}
ToIntFunction<String> parseToInt =
  s -> Integer.parseInt(s);
\end{lstlisting}

\pause{} We call a function that has the same type as the target function type, \lstinline{String} $->$ \lstinline{int}. Instead, we can use a \emph{function reference}: \pause{}

\begin{lstlisting}
ToIntFunction<String> parseToInt =
  Integer::parseInt;
\end{lstlisting}

This makes the code much more concise!
\end{frame}

\begin{frame}{Functional Interfaces Summary}
  The important points until here:

  \begin{itemize}
  \pause{} \item In Java 8, you can use \emph{lambda expressions} instead of inner classes.
  \pause{} \item You can define your own \emph{target function types} by defining interfaces.
  \pause{} \item Better: use pre-defined functional interfaces from \lstinline{java.util.function.*}
  \pause{} \item If you work on primitive data types, use specialized interfaces!
  \pause{} \item So what can we use these for?
  \end{itemize}
\end{frame}

\begin{frame}[fragile]{Java 8 Streams}
Streams in Java 8 are lazy collections of values:

\begin{lstlisting}
Stream<Student> students =
  Stream.of(new Student("A", 89),
            new Student("B", 44),
            new Student("C", 62));
\end{lstlisting}

\pause{} We can map over streams:

\begin{lstlisting}
students
  .map(s -> new Student(s.name, s.grade + 10));
\end{lstlisting}

\begin{itemize}
\pause{} \item \emph{Functional}: \lstinline{map} returns a new \lstinline{Stream<Student>} instance!
\pause{} \item \emph{Lazy}: the computation is not executed directly.
\pause{} \item Requires a \emph{terminal operation} to trigger computation.
\end{itemize}

\pause{}

\begin{lstlisting}
students
  .map(s -> new Student(s.name, s.grade + 10))
  .forEach(System.out::println);
\end{lstlisting}
\end{frame}

\begin{frame}[fragile]{Collector Classes}
  \begin{itemize}
  \item How can we compute the average grade of the class?
  \pause{} \item Using \lstinline{map} and \lstinline{count} takes $2n$ work and the stream can only be used once, so not an option!
  \pause{} \item Instead, use pre-defined \lstinline{Collector}:
  \end{itemize}

  \begin{lstlisting}
students
  .collect(Collectors
    .averagingInt((Student s) -> s.grade));
\end{lstlisting}

There are many different collectors already pre-defined!
\end{frame}

\begin{frame}[fragile]{What is Laziness?}
Deferring computation until value is requested.

\begin{lstlisting}
public class Lazy<T> implements Supplier<T> {
  private final Supplier<T> f;
  private volatile T t;

  public Lazy(Supplier<T> f) {
    this.f = f;
  }

  public T get() {
    if (t == null)
      t = f.get();
    return t;
  }}
\end{lstlisting}
\end{frame}

\begin{frame}[fragile]{Laziness in Action}
\begin{lstlisting}
Lazy<String> s = new Lazy<String>(() -> {
    System.out.println("Calling get()");
    return "someString";
  }); // Not yet computed!

String s1 = s.get(); // "Calling get()"
String s2 = s.get(); // Nothing printed.
\end{lstlisting}

\pause{} \vspace{1cm} \textbf{OBS:} If we use side-effects, this implementation is \textbf{not} thread-safe because \lstinline{Lazy::get} might be called twice.

\end{frame}

\begin{frame}[fragile]{How Does Laziness Help?}
Laziness allows the \lstinline{Stream} implementation to \emph{fuse} successive applications of \lstinline{map}. This is an in-place mapping over an array:

\begin{lstlisting}
  for (int i = 0; i < students.length; ++i)
    students[i] = f(students[i]);
  for (int i = 0; i < students.length; ++i)
    students[i] = g(students[i]);
\end{lstlisting}

\pause{} This is what loop fusion does:
\begin{lstlisting}
  for (int i = 0; i < students.length; ++i)
    students[i] = g(f(students[i]));
\end{lstlisting}

\pause{} You can do it manually -- we call it \emph{function composition}:
\begin{lstlisting}
  Function<A, B> f = ...;
  Function<B, C> g = ...;
  Function<A, C> h = f.andThen(g);
\end{lstlisting}
\end{frame}

\begin{frame}[fragile]{Computing Number of Primes}
For-loop:
\begin{lstlisting}
long count = 0;
for (long p = 0; p < n; ++p)
  if (isPrime(p)) ++count;
\end{lstlisting}

\pause{} Sequential stream:
\begin{lstlisting}
LongStream.range(0, n)
  .filter(p -> isPrime(p))
  .count();
\end{lstlisting}

\pause{} Parallel stream:
\begin{lstlisting}
LongStream.range(0, n)
  .parallel()
  .filter(p -> isPrime(p))
  .count();
\end{lstlisting}

\footnotesize{Modified from \emph{P. Sestoft, Java Precisely, third ed., The MIT Press, Mar. 2016}.}
\end{frame}

\begin{frame}{Java Streams Summary}
  Java streams are a great tool:
  \begin{itemize}
  \item Lazy, higher-order functional and declarative.
  \item Very well implemented, highly optimized.
  \item Easy to use when you are used to functional programming.
  \item Easy to parallelize (\lstinline{.parallel()}).
  \end{itemize}

  \pause{} But the downsides\dots{}
  \begin{itemize}
  \item Sometimes, performance is unpredictable because of laziness.
  \item OBS: Functions with side-effects will break!
  \end{itemize}
\end{frame}

\begin{frame}[fragile]{Building Streams}
\begin{lstlisting}
IntStream is = IntStream.of(2, 3, 5, 7, 11, 13);
IntStream nats = IntStream.iterate(0, x -> x+1);
\end{lstlisting}
\pause{} \begin{lstlisting}
String[] a = { "Haidian", "Gulou", ...};
Stream<String> beijingNeigbourhoods =
  Arrays.stream(a);
\end{lstlisting}
\pause{} \begin{lstlisting}
Collection<String> coll = ...;
Stream<String> countries =
  coll.stream();
\end{lstlisting}
\pause{} \begin{lstlisting}
Map<String,Integer> phoneNumbers = ...;
Stream<Map.Entry<String,Integer>> phones =
  phoneNumbers.entrySet().stream();
\end{lstlisting}

\footnotesize{Modified from \emph{P. Sestoft, Java Precisely, third ed., The MIT Press, Mar. 2016}.}
\end{frame}

\begin{frame}[fragile]{Take-Aways From This Lecture}
  We have seen:

  \begin{itemize}
  \pause{} \item Lambda expressions in Java 8.
    \begin{lstlisting}[basicstyle=\footnotesize\ttfamily]
Function<String, String> toUpperCase =
 s -> s.toUpperCase();
    \end{lstlisting}
  \pause{} \item Functional interfaces and how to define your own.
  \pause{} \item Using streams in Java 8.
    \begin{lstlisting}[basicstyle=\footnotesize\ttfamily]
students.
  map(s -> new Student(s.name, s.grade + 10));
    \end{lstlisting}
  \pause{} \item Using \emph{parallel} streams in Java 8.
    \begin{lstlisting}[basicstyle=\footnotesize\ttfamily]
students
  .parallel()
  .map(s -> new Student(s.name, s.grade + 10));
    \end{lstlisting}
  \end{itemize}
  \pause{}
  You will find them very useful!
\end{frame}

\begin{frame}{Projects}
  Please chose a project! You are allowed to work in groups of 2 -- 3 students.

  \begin{description}
  \pause{} \item[Typing an Untyped Racket Library] Find an untyped Racket library on \url{pkgs.racket-lang.org} and annotate it with types.
  \pause{} \item[Parallel Immutable Functional Data Structures] Extend the Rope data structure with more functions and parallelize them.
  \pause{} \item[A Tiny Math Language Implementation] Implement an interpreter and a type checker for a math language with Racket-style syntax.
  \pause{} \item[Your own idea?] If you have a personal project that you want to work on using Typed Racket or Java 8 Streams, you are very welcome to do so.
  \end{description}
\end{frame}

\begin{frame}[fragile]{Typing an Untyped Racket Library}
  \begin{itemize}
  \item \url{pkgs.racket-lang.org} is the official library repository for Racket.
  \pause{} \item Untyped code:
    \begin{lstlisting}[style=Racket]
(define (id x) x)
    \end{lstlisting}
  \pause{} \item Your task:
    \begin{lstlisting}[style=Racket]
(: id (All (A) (-> A A)))
(define (id x) x)
    \end{lstlisting}
  \pause{} \item Please e-mail me which package you want to work on, so that I can check it is not too hard for you.
  \end{itemize}
\end{frame}

\begin{frame}{Parallel Immutable Functional Data Structures}
  \begin{itemize}
  \item Last lecture you saw the Rope data structure for parallel programming.
  \pause{} \item Much more work to be done!
    \begin{itemize}
    \item \lstinline[style=Racket]{rope-mapreduce}
    \item \lstinline[style=Racket]{rope-reverse}
    \item \lstinline[style=Racket]{rope-forall}
    \item \lstinline[style=Racket]{rope-exists}
    \item \lstinline[style=Racket]{rope-scan}
    \end{itemize}
  \pause{} \item Parallelize functions that are not parallel.
  \end{itemize}
\end{frame}

\begin{frame}[fragile]{A Tiny Math Language Implementation}
  \begin{itemize}
  \item In Racket, you can easily implement other languages that look like Racket.
    \begin{lstlisting}[style=Racket]
(let (x (neg 1)) (* x x))
    \end{lstlisting}
  \pause{} \item You need not to implement the parser.
  \pause{} \item Tasks:
    \begin{itemize}
    \item Implement the interpreter.
    \item Implement a type checker.
    \item Maybe implement a compiler?
    \end{itemize}
  \end{itemize}
\end{frame}

\begin{frame}{Project Descriptions}
  The project descriptions are available on

  \begin{center}
    \url{github.com/fbie/parallel-functional-lectures}
  \end{center}

  Hand-in deadline is

  \begin{center}
    2016 -- 07 -- 13
  \end{center}

  Please write some text to explain your implementation and hand-in your code by e-mail to \texttt{fbie@itu.dk}.
\end{frame}

\begin{frame}{Questions?}
  \begin{center}
    Slides and code available at \url{github.com/fbie/parallel-functional-programming}.
  \end{center}

  \pause{}

  \begin{center}
    Any questions? E-mail me at \texttt{fbie@itu.dk}.
  \end{center}

  \pause{}

  \begin{center}
    \textbf{Thank you for your attention!}
  \end{center}
\end{frame}
\end{document}
